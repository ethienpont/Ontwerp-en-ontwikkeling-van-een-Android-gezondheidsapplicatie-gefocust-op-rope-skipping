\chapter*{Introductie}

\section*{Probleemstelling}
Sport en beweging wint steeds aan populariteit in onze samenleving. Meer mensen gebruiken dan ook mobiele applicaties om hun sportprestaties en algemene gezondheid te verbeteren.  

Huidige gezondheidsapplicaties geven echter vaak enkel statische aanbevelingen. Ze houden geen rekening met de persoonlijke vooruitgang van de gebruiker, zijn of haar conditie en fysieke capaciteiten.  Gebruikers moeten zelf hun gewenste doelstellingen opgeven. De applicatie zal dan, gebaseerd op deze parameters, bijhorende aanbevelingen geven. 

Rope skipping is een sport die voor een goede conditie zorgt en weinig plaats of tijd vergt. De sport is dus uitermate geschikt om mensen aan te zetten tot meer beweging. Een diepgaande opvolging van deze activiteit werd nog niet eerder op punt gezet.
 
Indien de gebruiker zijn/haar fysieke grenzen overschrijdt, omwille van vrijwillige forcering of ongezonde omstandigheden, wordt dit meestal niet gemeld. Hierdoor kunnen langdurige klachten ontstaan, wat de gezondheid zeker niet ten goede komt. Een classificatie van de gebruikerscontext in termen van gezondheidstoestand is noodzakelijk. 
 
\section*{Doelstelling}
Het doel van deze thesis is het ontwikkelen van een mobiele applicatie die gebruikers een gezonde levensstijl aanleert door persoonlijke aanbevelingen te produceren. Deze aanbevelingen zullen duidelijk maken wat de gebruiker moet doen om in optimale gezondheid te blijven. De inhoud hiervan is gebaseerd op de fysieke activiteiten van de gebruiker en wat zijn/haar huidige conditie toelaat (gepersonaliseerd). Dit onderzoek zal zich focussen op rope skipping aangezien dit de doelstelling van de thesis goed verwezenlijkt. 
 
De thesis zal een antwoord bieden op volgende onderzoeksvraag: “Hoe kan vanuit ruwe data (hartslag, beweging) een gepersonaliseerd bewegingspatroon ontwikkeld worden?” 

Het probleem van statische, niet gepersonaliseerde aanbevelingen wordt aangepakt door een op maat gemaakt bewegingspatroon te creëren voor iedere gebruiker. Via algoritmen die ruwe data van sensoren (hartslag, beweging) omzetten in specifieke activiteiten of inspanningsniveaus zal dit ontwikkeld worden.  
 
Ook zal er aandacht besteed worden aan bepaalde deelonderzoeksvragen. Een eerste van deze deelonderzoeksvragen luidt als volgt: “Is het mogelijk om specifieke rope skipping bewegingen te herkennen op basis van accelerometer data?". Aangezien accelerometer een zeer energiezuinige sensor is, focust deze thesis zich hoofdzakelijk op deze data. Via bewegingsherkenning kan gedetecteerd worden welke activiteiten de gebruiker verkiest. Hierdoor wordt het systeem persoonlijker. Ook zal door middel van foutdetectie een aanmoedigingsmechanisme ontwikkeld worden. 
 
Een tweede deelonderzoeksvraag omvat: “Wanneer wordt intensief bewegen ongezond?” Een classificatie van de gebruikerscontext in termen van gezondheidstoestand kan bekomen worden door constante monitoring van de hartslag. Via deze data kunnen ongewone patronen gedetecteerd worden. Hieruit kan afgeleid worden of de fysieke toestand van de gebruiker verdere intensieve beweging toelaat. 
 
 
