\textbf{\Huge Ontwerp en ontwikkeling van een gezondheidsaanbevelingssysteem gericht op rope skipping} \\

Promotoren: prof. dr. ir. Toon De Pessemier, prof. dr. ir. Luc Martens \\
Masterproef ingediend tot het behalen van de academische graad van Master of Science in de industriële wetenschappen: informatica \\
Academiejaar 2019-2020
\\\\
\textbf{\LARGE Abstract} \\\\
Fysieke activiteit is broodnodig in onze sedentaire samenleving. Zonder enige vorm van persoonlijke coaching is dit echter niet efficiënt te realiseren. Tegenwoordig heeft iedereen wel een soort mobiel toestel op zak. Dit is een bron van mogelijkheden op vlak van fysieke activiteit coaching. Een smartwatch geeft hier nog een extra dimensie aan door de fysieke activiteit rechtstreeks te monitoren op basis van hartslagsensoren. Deze data maakt het mogelijk om ook persoonlijke aanbevelingen te produceren.
Één van de meest gebruikte argumenten in het voordeel van te weinig beweging is het gebrek aan tijd. Hiervoor is \textit{rope skipping} de ideale oplossing. Deze sport zorgt namelijk voor optimale conditietraining waardoor gebruikers efficiënt bewegen. Ook kan deze sport eender waar uitgeoefend worden mits enige plaats. Qua bewegingsherkenning van specifieke rope skipping bewegingen werd nog bitter weinig onderzoek uitgevoerd. Door de bewegingen te herkennen en eventuele tekortkomingen te detecteren, kan echter voor extra aanmoediging gezorgd worden.
Deze paper beschrijft een android applicatie ontwikkeld om de gebruiker een gezond bewegingspatroon aan te leren met toevoeging van het extra element rope skipping. 
In een eerste deel wordt bestaande literatuur bekeken met betrekking tot bewegingsherkenning, bepalen van inspanningsniveaus, doel voorspelling en aanbevelingssystemen.
Een tweede deel gaat dieper in op de gebruikte technologieën. Een goed inzicht in het materiaal/de technologieën waarmee gewerkt wordt is namelijk vereist. 
Vervolgens wordt meer verteld over het bewegingsherkenning proces. Door verzameling van data afkomstig van verschillende proefpersonen wordt een model ontwikkeld. Dit model is in staat om vijf rope skipping bewegingen te classificeren.
In een laatste deel wordt de gezondheidsapplicatie toegelicht. Deze applicatie gaat, gebaseerd op het inspanningsniveau bij de verschillende rope skipping bewegingen, aanbevelingen genereren. Het inspanningsniveau wordt bepaald aan de hand van de MET metriek. Het aantal METs is afhankelijk van de tijd die in een bepaalde hartslag zone doorgebracht werd. Aanbevelingen worden berekend aan de hand van enerzijds de frequentie van uitvoering en het gemiddeld aantal METs verbruikt per minuut per beweging, anderzijds de hoeveelheid gemaakte fouten. Door te werken met een doel wordt een bovengrens gecreëerd voor het aantal aanbevelingen. Dit doel wordt bepaald door historische data tot 10 weken in het verleden te bekijken en hier het gemiddelde van te nemen.\\\\

\noindent
\textbf{Trefwoorden: rope skipping, gezondheidsapplicatie, wear OS, android, recommender system, machine learning, neuraal netwerk}

